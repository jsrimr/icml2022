%%%%%%%% ICML 2022 EXAMPLE LATEX SUBMISSION FILE %%%%%%%%%%%%%%%%%

\documentclass[nohyperref]{article}

% Recommended, but optional, packages for figures and better typesetting:
\usepackage{microtype}
\usepackage{graphicx}
\usepackage{subfigure}
\usepackage{booktabs} % for professional tables

% hyperref makes hyperlinks in the resulting PDF.
% If your build breaks (sometimes temporarily if a hyperlink spans a page)
% please comment out the following usepackage line and replace
% \usepackage{icml2022} with \usepackage[nohyperref]{icml2022} above.
\usepackage{hyperref}


% Attempt to make hyperref and algorithmic work together better:
\newcommand{\theHalgorithm}{\arabic{algorithm}}

% Use the following line for the initial blind version submitted for review:
% \usepackage{icml2022}
\usepackage{pre-training}

% If accepted, instead use the following line for the camera-ready submission:
% \usepackage[accepted]{icml2022}

% For theorems and such
\usepackage{amsmath}
\usepackage{amssymb}
\usepackage{mathtools}
\usepackage{amsthm}

% if you use cleveref..
\usepackage[capitalize,noabbrev]{cleveref}

% custom
\usepackage{tabularx} % in the preamble
\usepackage{lipsum}
\usepackage{tabu}
\usepackage{booktabs}% for better rules in the table

%%%%%%%%%%%%%%%%%%%%%%%%%%%%%%%%
% THEOREMS
%%%%%%%%%%%%%%%%%%%%%%%%%%%%%%%%
\theoremstyle{plain}
\newtheorem{theorem}{Theorem}[section]
\newtheorem{proposition}[theorem]{Proposition}
\newtheorem{lemma}[theorem]{Lemma}
\newtheorem{corollary}[theorem]{Corollary}
\theoremstyle{definition}
\newtheorem{definition}[theorem]{Definition}
\newtheorem{assumption}[theorem]{Assumption}
\theoremstyle{remark}
\newtheorem{remark}[theorem]{Remark}

% Todonotes is useful during development; simply uncomment the next line
%    and comment out the line below the next line to turn off comments
%\usepackage[disable,textsize=tiny]{todonotes}
\usepackage[textsize=tiny]{todonotes}


% The \icmltitle you define below is probably too long as a header.
% Therefore, a short form for the running title is supplied here:
\icmltitlerunning{Submission and Formatting Instructions for ICML 2022}

\begin{document}

\twocolumn[
\icmltitle{Sample-Efficient finetuning by Interpreting skill as perspective}

% It is OKAY to include author information, even for blind
% submissions: the style file will automatically remove it for you
% unless you've provided the [accepted] option to the icml2022
% package.

% List of affiliations: The first argument should be a (short)
% identifier you will use later to specify author affiliations
% Academic affiliations should list Department, University, City, Region, Country
% Industry affiliations should list Company, City, Region, Country

% You can specify symbols, otherwise they are numbered in order.
% Ideally, you should not use this facility. Affiliations will be numbered
% in order of appearance and this is the preferred way.
\icmlsetsymbol{equal}{*}

\begin{icmlauthorlist}
\icmlauthor{Firstname1 Lastname1}{equal,yyy}
\icmlauthor{Firstname2 Lastname2}{equal,yyy,comp}
\icmlauthor{Firstname3 Lastname3}{comp}
\icmlauthor{Firstname4 Lastname4}{sch}
\icmlauthor{Firstname5 Lastname5}{yyy}
\icmlauthor{Firstname6 Lastname6}{sch,yyy,comp}
\icmlauthor{Firstname7 Lastname7}{comp}
%\icmlauthor{}{sch}
\icmlauthor{Firstname8 Lastname8}{sch}
\icmlauthor{Firstname8 Lastname8}{yyy,comp}
%\icmlauthor{}{sch}
%\icmlauthor{}{sch}
\end{icmlauthorlist}

\icmlaffiliation{yyy}{Department of XXX, University of YYY, Location, Country}
\icmlaffiliation{comp}{Company Name, Location, Country}
\icmlaffiliation{sch}{School of ZZZ, Institute of WWW, Location, Country}

\icmlcorrespondingauthor{Firstname1 Lastname1}{first1.last1@xxx.edu}
\icmlcorrespondingauthor{Firstname2 Lastname2}{first2.last2@www.uk}

% You may provide any keywords that you
% find helpful for describing your paper; these are used to populate
% the "keywords" metadata in the PDF but will not be shown in the document
\icmlkeywords{Machine Learning, ICML}

\vskip 0.3in
]

% this must go after the closing bracket ] following \twocolumn[ ...

% This command actually creates the footnote in the first column
% listing the affiliations and the copyright notice.
% The command takes one argument, which is text to display at the start of the footnote.
% The \icmlEqualContribution command is standard text for equal contribution.
% Remove it (just {}) if you do not need this facility.

%\printAffiliationsAndNotice{}  % leave blank if no need to mention equal contribution
\printAffiliationsAndNotice{\icmlEqualContribution} % otherwise use the standard text.

\begin{abstract}
Skill is an attempt to integrate pretrain-finetuning method in reinforcement learning.
During pretraining, the agent learns some useful and distinct policies without reward, and use this policy at finetuning to get fast and better performance than training from scratch.
However, there's few attempts to combine skills optimally at finetuning.
In this paper, we propose two skill combining methods which are sample-efficient and shows good performance.
First method is a state-agnostic super sample efficient method achieving higher reward than baseline methods adding only tens of parameters.
Second method is a state-aware skill combining method which is a slower-learner than first method but shows better performance at the end.
We improve second method's sample efficiency by using pretrained DIAYN module as a skill combining module.
Code is available at $https://github.com/some-private$.

\end{abstract}

%%%%%%%%% BODY TEXT
\section{introduction}
\label{submission}

\begin{figure}[ht]
  \vskip 0.2in
  \begin{center}
  \centerline{\includegraphics[width=\columnwidth]{Figures/figure_diagram.pdf}}
  \caption{Overview of methods to mix a set of learned skills in fine-tuning phase. It is optional whether the current state affects how the skills combine together.}
  \label{overview}
  \end{center}
  \vskip -0.2in
\end{figure}

Unsupervised learning for deep reinforcement learning (DRL) is challenging in the sense that an agent should interact with the world and collect data to obtain useful information without explicit reward.
In the pretraining step, an agent gathers information from the environment and learns representations of visited states, dynamics, and behaviors.
When the downstream tasks are given, the agent has to quickly adapt to them leveraging those representations.
Recent works have shown that skill discovery is a promising way to learn diverse behaviors which potentially help the agent achieve a goal \cite{salge2014empowerment, gregor2016variational, eysenbach2018diversity}.

Although the agent could learn diverse skills, it is not obvious at a glance how the agent manages them to solve a downstream task since the agent has to efficiently and effectively handle multiple policies while the agent has just one policy to control in a typical reinforcement learning framework.
DIAYN\cite{eysenbach2018diversity} suggested testing all skills in downstream tasks. However, this is very sample-inefficient.
The recent work \cite{laskin2021urlb} which evaluated a variety of unsupervised RL algorithms in locomotion environments, uniformly sampled skill for every episdode even at the finetuning phase.
This leads to significant performance degradation and naturally DIAYN showed poor performance compared to other algorithms according to their work.

Instead of exhaustive skill search nor naive skill sampling, we propose two skill searching methodsin a sample efficient manner.
Both method evaluates the importance of each skill, but the first evaluates the importance of a skill in a state-agnostic way
while the second evaluates the importance of a skill in a state-aware way.
The first method adds tens of parameters($\sim skilldim$) to help agent learn the importance of each skill. With these simple adjustments, the agent can create a policy that considers all skills and achieve good performance in a sample efficient way.
The second method is to add a neural net module that evaluates  skill importance differently according to the input state.
This method has a smoother running curve than the first method, but the final performance is better.
What is noteworthy here is that we use the weight of DIAYN module from the pretraining phase.
During pretraining, the DIAYN module predicted which skill was used when given the state. We thought that this state to skill mapping learning process would be helpful when finetuning.
As expected, with these pretrain weights, the state-aware importance predictor achieved higher performance in a more sample efficient manner.

Moreover, we particularly were curious on whether the agent can acquire the ability to use a combination of several skills rather than just using one skill.
We often see cases where smart communities see the same event from different perspectives and benefit more through combination of perspectives.
As illustrated in \cref*{fig:skill_as_perspective},
when using different skill, different skill weigth vector is used while the transformed state vector remains same regardless of which skill is used.
We interpret this skill weight vector as agent's perspectives. The agent could interprete incoming state differently by adding different skill weight vector.
What we are curious about is whether combining these skills which is adding several skill weight vector to the transformed state vector would happen when the agent is allowed to.
Unfortunately, even when the model performed well, we couldn't observe combination of skills.
Perhaps this is the task was not difficult enough to require a combination of skills.
However, if the task becomes difficult, we expect skill combination will be needed and we leave this left as further work.
\begin{figure}[ht]
  \vskip 0.2in
  \begin{center}
  \centerline{\includegraphics[width=\columnwidth]{Figures/state-skill-decomposition.png}}
  \caption{The Agent is using second skill weight vector to interprete the input state. We inspect combining first and third skill weight vector could contribute to better learning.}
  \label{fig:skill_as_perspective}
  \end{center}
  \vskip -0.2in
\end{figure}

% Our main contribution is to emphasize the importance of fine-tuning in skill discovery by showing the finetuning method can affect the final score a lot for each task.
% We also propose a simple but powerful finetune method utilizing every learned skill.

Our contribution is to introduce a method in which the agent can determine the importance of each skill by itself.
Two methods are presented, a state-agnostic method that can quickly achieve performance above a certain level, and a state-aware method that is slightly slower but more powerful.
In particular, in the second method, the sample-efficiency was increased by using the weight of the DIAYN discriminator used during pretraining, and the final performance was further improved.

\section{Introduction} expert data to train a state encoder through an auxiliary classifier, which tries to distinguish expert-visited states While autonomous learning of diverse and complex behavfrom random states. We then use the encoder to project iors is challenging, significant progress has been made using the state space into a latent embedding that preserves infordeep reinforcement learning (DRL). The progress has been mation that makes expert-visited states recognizable. This method extends readily to other mechanisms of learned stateaccelerated by the powerful representational learning of deep neural networks (LeCun et al. 2015 ), and the scalaprojections and different skill discovery algorithms. Crubility and efficiency of RL algorithms (Mnih et al. 2015 . cially, our method requires only samples of expert-visited Schulman et al. 2017: Haarnoja et al. 2018 . Lillicrap et al. ). states, which can easily be obtained from any reference However, DRL still involves an externally designed reward policy, for example expert demonstrations. function that guides learning and exploration. Manually The key contribution of this paper is a simple method engineering such a reward function is a complex task that for learning a parameterized state projection that guides requires significant domain knowledge in such a way that skill discovery towards a substructure of the observation hinders autonomy and adoption of RL. Prior works have prospace. We demonstrate the flexibility of our state-projection posed using unsupervised skill discovery to alleviate these method and how it can be used with the skill-discovery challenges by using empowerment as an intrinsic motivation objective. We also present empirical results that show the to explore and acquire abilities (Salge et al. 2014, Gregor performance of our method in various locomotion tasks. ${ }^{*}$ Equal contribution ${ }^{1}$ Department of Computer Science, Norwegian University of Science and Technology ${ }^{2}$ Department of Information Security and Communication Technology, Norwegian

\section{Related Work} University of Science and Technology. Correspondence to: Even Klemsdal <even.klemsdal@ntnu.no> Unsupervised reinforcement learning aims at learning diverse behaviours in a task-agnostic fashion without guidance Accepted by the ICML 2021 workshop on Unsupervised Reinforcement Learning, PMLR 139,2021 . Copyright 2021 by the from an extrinsic reward function (Jaderberg et al. 2016). author(s). This can be accomplished through learning with an intrin- sic reward such as curiosity (Oudeyer \& Kaplan 2009 or tion: $\mathcal{M}=(\mathcal{S}, \mathcal{A}, \mathcal{P})$, where $\mathcal{S}$ is the state space, $\mathcal{A}$ is the empowerment (Salge et al. 2014). The notion of curiosity action space, and $\mathcal{P}: \mathcal{S} \times \mathcal{S} \times \mathcal{A} \rightarrow[0, \infty)$ is the tranhas been utilized for exploration by using predictive models sition probability density function. The RL agent learns of the observation space and providing a higher intrinsic a skill-conditioned policy $\pi_{\vartheta}(a \mid s, z)$, where the skill $z$ is reward for visiting unexplored trajectories (Pathak et al. sampled from some distribution $p(z)$. A skill, or option 2017). Empowerment addresses maximizing an agent's con(as first introduced in (Sutton \& Barto, 2018), is a temtrol over the environment by exploring states with maximal poral abstraction of a course of actions that extends over intrinsic options (skills). many time steps. We will also consider the informationtheoretic notion of mutual information between states and Several approaches have been proposed in the literature to skills $\mathcal{I}(S ; Z)=\mathcal{H}(Z)-\mathcal{H}(Z \mid S)$, where $\mathcal{H}(\cdot)$ is the Shanutilize empowerment for skill discovery in unsupervised RL. non entropy. Gregor et al. Gregor et al. 2016 developed an algorithm that learns intrinsic skill embedding and used generalization to discover new goals. They used the mutual information

\subsection{Skill Discovery Objective} between skills and final states as the training objective and The overall goal of skill discovery is to find a policy $\pi_{\vartheta}$ cahence used a discriminator to distinguish between different pable of carrying out different tasks that are learned without skills. Eysenbach et al. (Eysenbach et al. 2018) used mutual information between skills and states as an objective extrinsic supervision for each type of behavior. We conwhile using a fixed embedding distribution of skills. Adsider policies of the form $\pi_{\vartheta}(a \mid s, z)$ that specify different distributions over actions depending on which skill $z$ they ditionally, they used a maximum-entropy policy (Haarnoja are conditioned on. Although this general framework does et al. 2018 to produce stochastic skills. However, most of the previous approaches assume a state distribution induced not constrain how $z$ should be represented, we define it as a by the policy itself, resulting in a premature commitment discrete variable since it has been empirically shown to perform better than continuous alternatives (Eysenbach et al. to already discovered skills. Campos et al. Campos et al. 2018 . 2020 used a fixed uniform distribution over states to break the dependency between the state distribution and the policy. We follow the framework proposed by the "diversity is all you need" (DIAYN) algorithm (Eysenbach et al. 2018), Certain prior work has addressed the challenge of complex and high dimensional state space by constraining the in which skills are learned by defining an intrinsic reward that promotes diversity. Intuitively, each skill should make skill-discovery in a subset of the state space. Sharma et al. the agent visit a unique section of the state space. This (Sharma et al. 2019 l learned predictable skills by training a can be expressed as maximising the mutual information skill-conditioned dynamic model instead of a discriminator to model specific behaviour in a subset of the state space. $\mathcal{I}(S ; Z)$ of the state visitation distributions for different Eysenbach et al. (Eysenbach et al. 2018 ) proposed incorskills (Salge et al. 2014). To ensure that the visited areas of the state space are spaced sufficiently far apart, we use a soft porating prior knowledge by conditioning the discriminator policy that maximises the entropy of the action distribution. on a subset of the state space using a hand-crafted and a task-specific transformation. Our work addresses this chalFormally, we maximize the following objective function: lenge by guiding the skill discovery towards the subset of expert-visited states. In contrast to inverse reinforcement learning, (Fu et al. 2018 , we do not explicitly infer the extrinsic reward. Crucially, we do not try to learn the expert policy directly in contrast to behaviour cloning or imitation

$$
\begin{aligned}
\mathcal{F}(\theta) & \triangleq I(S ; Z)+\mathcal{H}(A \mid S)-I(A ; Z \mid S) \\
&=\mathcal{H}[A \mid S, Z]-\mathcal{H}[Z \mid S]+\mathcal{H}[Z]
\end{aligned}
$$
learning 2011 . Our proposed method resembles the algorithm proposed by Li et al. (Li et al. 2020 ) in which they used a Bayesian classifier that estimates the probability of successful outcome states, resulting in a more The first term $\mathcal{H}[A \mid S, Z]$ means that the policy should act as task-directed exploration. However, their algorithm does randomly as possible and can be optimized by maximizing not optimize the mutual information; hence it does not learn the policy's entropy. The second term $-\mathcal{H}[Z \mid S]$ dictates diverse skills via the discriminability objective. that each visited state should (ideally) identify the current skill. The third term is the entropy of the skill distribution, which can be maximized by deliberately sampling skills

\section{Preliminaries} from a uniform distribution during training. Unfortunately, $-\mathcal{H}[Z \mid S]$ requires knowledge about $p(z \mid s)$, which is not In this paper, we formalize the problem of skill discovery as readily available. Consequently, we approximate the true a Markov decision process (MDP) without a reward funcdistribution by training a classifier $q_{\phi}(z \mid s)$, leading to a 

$$
\begin{aligned}
&\text { lower bound: } \\
&\begin{aligned}
\mathcal{F}(\theta) &=\mathcal{H}[A \mid S, Z]+\mathbb{E}_{p}[\log p(z \mid s)]-\mathbb{E}_{p}[\log p(z)] \\
& \geq \mathcal{H}[A \mid S, Z]+\mathbb{E}_{p}\left[\log q_{\phi}(z \mid s)\right]-\mathbb{E}_{p}[\log p(z)]
\end{aligned}
\end{aligned}
$$

The lower bound follows from the non-negative property of the Kullback-Leibler divergence $D_{K L}(p \| q)=$ $\mathbb{E}_{p}[\log p(z \mid s)-\log q(z \mid s)] \geq 0$, which can be rearranged to $\mathbb{E}_{p}[\log p(z \mid s)] \geq \mathbb{E}_{p}[\log q(z \mid s)]$ Agakov 2004$)$.

The classifier $q_{\phi}$ is fitted throughout training with maximum likelihood estimation over the sampled states and active skills. This leads to a scenario where the policy is rolled out for a (uniformly) sampled skill, and the classifier is trained to detect the skill based on the states that were visited. The policy is given a reward proportional to how well the classifier could detect the skill in each state. In the end, this should make the policy favor visiting disjoint sets of states for each skill, leading to a cooperative game between $q_{\phi}$ and $\pi_{\vartheta}$.

improve the skills discovered for a navigation task, we aim to learn a parameterized $f_{\chi}$ by using expert data.

\subsection{Limitations of existing methods}

A major challenge that arises when maximizing the objec-

\subsection{State Space Projections} tive in Equation 2 particularly in applications with highdimensional spaces, is that it becomes trivial for each skill

We consider linear projections of continuous factored state representations on the form $f_{\chi}: \mathbb{R}^{|S|} \rightarrow \mathbb{R}^{|E|}$ with $e=$ to find a sub-region of the state space where it is easy to be recognised by $q_{\phi}$. In preliminary experiments, we ob$f_{\chi}(s)=\chi s, \chi \in \mathbb{R}^{|E| \times|S|}$, and $|E|<|S|$. In principle, the idea should apply to more complex mappings, such as a served that the existing methods discovered behaviours that multi-layer perceptron. However, we want to limit the scope covered small parts of the state space. For the HalfCheetah of skill discovery to a hyperplane within the original state environment (Brockman et al. 2016) this resulted in many space. skills generating different types of static poses (see Figure 1 and not many skills exhibiting "interesting" behaviour

For the same reason, we also omit any non-linearities in the such as locomotion. encoder. Squeezing the output through a Sigmoidal function would limit discriminability at the (potentially interesting)

Optimising for $\mathcal{H}[A \mid S, Z]$ should mitigate this issue to some extent. Increasing the policy's entropy incentivises the skills extremes of the encoding. Similarly, a ReLU function would effectively eliminate all exploration along the negative dito progressively visit regions of the state space that are so rection of $f_{\chi_{i}}$. In summary, the objective of the DIAYN far apart that not even highly stochastic actions will cause skill classifier becomes: them to overlap accidentally. However, it has been shown that mutual information based algorithms have difficulties spreading out to novel states due to low values of $\log q_{\phi}(z \mid s)$

$$
\max _{\phi} \mathbb{E}\left[\log q_{\phi}(z \mid e)\right]
$$
for out-of-sample states (Campos et al. 2020 .

\section{Proposed Method}

We learn the parameters $\chi$ for the projection through an auxiliary discriminative objective. Specifically, a binary The main idea of our approach is to focus the skill discovclassifier $h_{\psi}: \mathbb{R}^{|\mathbb{E}|} \rightarrow\{0,1\}$ is trained to predict whether ery towards certain parts of the state space by using expert an (encoded) state was sampled from the marginal state data as a prior. The DIAYN algorithm can be biased tovisitation distribution of a random policy $\pi_{\text {rand }}$ or from the wards a user-specified part of the state space by changing distribution of a reference (expert) policy $\pi^{*}$. Let $x \sim \mathcal{D}$ the discriminator to maximize $\mathbb{E}\left[\log q_{\phi}(z \mid f(s))\right]$, where $f$ denote whether a state $s$ was visited by the reference policy represents some transformation of the state space (Eysenor not (in dataset $\mathcal{D}$ ), then the parameters of $f_{\chi}$ are obtained bach et al. 2018 . Instead of using a hand-crafted $f$ to through joint pretraining with $h_{\psi}$ by maximizing the log 

$$
\begin{aligned}
&\text { likelihood over } \mathcal{D} \text { : } \\
&\max _{\chi, \psi} \mathbb{E}_{x, s \sim \mathcal{D}}\left[x \log h_{\psi}(x \mid e)+(1-x) \log \left(1-h_{\psi}(x \mid e)\right)\right]
\end{aligned}
$$

same quantity. This allows us to sample differentiable actions and climb the gradient of the minimum of the two Q-functions (DDPG-style update (Lillicrap et al.)), giving us this objective:

$$
J\left(\pi_{\vartheta}\right)=\min _{i \in\{1,2\}} Q_{\theta}^{i}\left(\pi_{\vartheta}(a \mid s)\right)+\alpha \mathcal{H}\left(\pi_{\vartheta}(a \mid s)\right)
$$

where the dataset $\mathcal{D}$ is collected prior to training the main RL algorithm. The first half (random samples) are collected Like in DIAYN, we also use a Squashed Gaussian Mixture by rolling out $\pi_{\text {rand }}$ whereas the second half (reference samModel to promote diverse behaviour. ples) are collected by rolling out $\pi^{*}$. After the objective in Equation 4 is optimized, the discriminator $h_{\psi}$ is discarded Figure 2 illustrates the training process of the proposed and the projection encoding $f_{\chi}(s)$ is extracted to be used expert-guided skill discovery. First, we train the encoder for the objective in Equation 3 Analogous to autoencoders $f_{\chi}$ jointly with the auxiliary classifier $h_{\psi}$ using the external (Hinton \& Salakhutdinov 2006), the idea is that the emdataset $\mathcal{D}$. Secondly, we train the agent using an offline beddings produced by $f_{\chi}(s)$ should now contain a more policy algorithm (SAC), in which the agent samples a skill compact representation of the state space without collapsing $z \sim p(z)$, and then interacts with the environment by taking the dimensions that make "interesting" behaviour stand out. action $a_{t}$ according the skill-conditioned policy $\pi_{\vartheta}\left(a_{t} \mid s_{t}, z\right)$. The environment, then, transits to a new state according to While the use of a reference data changes our approach the transition probability $s_{t+1} \sim p\left(s_{t+1} \mid s_{t}, a_{t}\right)$. We add from a strictly unsupervised skill discovery algorithm, the this transition $\left(s_{z}, z, a_{t}, s_{t+1}\right)$ to the replay buffer $\mathcal{B}$. Simuldiscriminative objective in equation 4 resembles the objectaneously, the policy is updated by sampling a mini-batch tives used in adversarial inverse reinforcement learning (e.g. from the replay buffer $\mathcal{M} \sim \mathcal{B}$, then encoding the next states (Fu et al. 2018) ). However, it differs in that it makes no $e_{t+1}=f_{\chi}\left(s_{t+1}\right)$ and passing them through the discrimiattempts at matching the behaviour of a reference policy nator $q_{\phi}\left(z \mid e_{t+1}\right.$ to get the intrinsic reward. This reward is as it is used only as a prior for simplifying the state space. used by the Q-functions $Q_{\theta}\left(s_{t}, a_{t}, z\right)$ to minimize the soft This approach could also be used with samples from sevBellman residual and update the policy. A pseudocode for eral different reference policies with substantially different the proposed approach can be found in the supplementary marginal state distributions. As long as their variation can material be explained sufficiently without full use of the entire state space, a projection should simplify skill discovery.

\subsection{Implementation}

For learning diverse skills, we use DIAYN as a basis framework. DIAYN uses the Soft Actor-Critic (SAC) algorithm (Haarnoja et al. 2018 ) that is optimized using policy gradient style updates in contrast to the reparameterized version (DDPG style updates (Lillicrap et al.)). They also use a Squashed Gaussian Mixture Model to represent the policy $a \sim \pi_{\vartheta}=\tanh G M M\left(\mu_{\vartheta}(s), \sigma_{\vartheta}(s)\right)$. The learning objective is to maximize the mutual information between the state
and skill $I(S ; Z)$. This objective is optimized by replacing Figure 2: Framework for training expert-guided skill discovery. the task rewards with a pseudo-reward Green arrows show the training of the encoder, red arrows show the agent-environment interaction, while the blue arrows show

$$
r_{z}(s, a) \triangleq \log p_{\phi}(z \mid s)-\log p(z)
$$
the interactions for the offline policy training. Note there is no gradients through the encoder in the offline training. where $q_{\phi}$ is trained to discriminate between skills and $\mathrm{p}(\mathrm{z})$ is the fixed uniform prior over skills (Eysenbach et al. 2018 .

\section{Experiments} A skill is sampled from $z \sim p(z)$ and used throughout a full episode. In our experimental evaluation, we aim to demonstrate the impact of our approach of restricting skill discovery to a In contrast to DIAYN, we use two Q-functions $Q_{\theta}^{1}(s, a)$ projection subspace. We verify our method on both point$\& Q_{\theta}^{2}(s, a)$ where both $\mathrm{Q}$-functions attempt to predict the mazes and continuous control locomotion tasks. All the 



Figure 6: Impact of state-space projection illustrated as distribution of displacements along the locomotion axis used to calculate the extrinsic reward of the environments. By introducing a projection step, the skill search gets focused towards locomotion skills, resulting in a larger spread of displacements.

Table 1: Summary statistics for displacement (along the locomotion axis used to calculate the extrinsic reward of the environments) across the 50 skills learned. The values to the right of $\pm$ indicate standard deviation across 5 seeded runs.

ple 10 trajectories of length 1000 with fairly high returns at the environment's rewards. However, the environment (Ant: $5063.9 \pm 469.8$, Cheetah: $10656.4 \pm 673.5$, Hopper: reward also includes terms for energy expenditure, staying $3348.0 \pm 316.0$ ). The DIAYN algorithm is otherwise identialive (for Ant/Hopper), and collisions (Ant), which would cal to (Eysenbach et al. 2018 ) in terms of hyperparameters; obscure the results. Figure 6 shows the displacement distri$Q, \pi, q_{\phi}$, and $h_{\psi}$ use MLP architectures with 2 hidden laybution of the 50 skills across all runs. The same information ers of width 300 , the entropy bonus weight $\alpha$ is set to $0.1$, is summarized numerically in Table 1 and the number of skills is set to 50 . We limit each skilldiscovery run to $2.5$ million environment interactions but For a qualitative evaluation, we have also composed a video repeat each experiment 5 times with different random seeds with every skill across all runs (including training of SAC agents for $\pi^{*}$ ).

For quantitative evaluation, we look at the displacement

\section{Discussion} along the target locomotion axis for the extrinsic objective. In our approach, we would expect to observe skills that cover For HalfCheetah and Hopper, the runs with state encoding this axis well, i.e., skills that run forward and backward at (+ ENC(3|5)) exhibit a substantially larger spread than the baseline. The best forward-moving cheetah skill moves 178 different speeds. To test this, we roll out each skill deterministically $\left.\right|^{2}$ record its movement over 1000 time steps (or units forward ( $=3311$ environment return), and the best until it reaches a terminal state) and observe the inter-skill backwards-moving cheetah skill moves 186 units backwards spread. A similar assessment is possible by only looking $(=-4025$ environment return $)$. For the hopper environment, the best forward-moving skill manages to jump 20 ${ }^{2}$ Deterministic sampling from our GMM-based policy implies taking the mean of the component with the highest mixture proba- ${ }^{3}$ Video of skills: https://www youtube.com/watch? bility. $\mathrm{V}=\mathrm{X \times} 7$ RVNmv1 $\mathrm{tY}$ units forward, which corresponds to an environment reward of 3268 , which is on the same level as the reference data used to fit its encoder.

The results in the Ant environment are less impressive. There is hardly any difference in how the displacements are distributed for the three approaches, and the total movement is almost negligible. For reference, a good Ant agent trained against the extrinsic reward should obtain displacements in the $100 \mathrm{~s}$ when evaluated over the same trajectory horizon.

Looking at the generated Ant behaviour, we found that the skills produced with encoders typically moved even less than those generated by the baseline. This is not because it
is impossible to generate a linear projection that promotes Figure 7: Embedding weights for a linear projection of the Ant locomotion at various speeds, as the state representation of state space down to $\mathbb{R}^{3}$. Only weights for the first 27 (standardized) all three problems contains a feature for linear velocity along state features are visualized since the remaining 84 are always the target direction. Moreover, the skill classifier does reach zero. The feature that corresponds to (whole body) velocity in the a high accuracy (some breaking $90 \%$ ), so the algorithm mansame direction as the main environment objective is highlighted in orange. ages to find distinguishable skills. We, therefore, suspect that the procedure used to fit the encoder is insufficient for this environment. While it does pick up on linear velocity, it

\section{Conclusion} also picks up on several other features from the state space, which might have made it easier for the algorithm to make In this work, we propose a data-driven approach for guidthe skills distinguishable. ing skill discovery towards learning useful behaviors in complex and high-dimensional spaces. Using examples of To better understand the results of the Ant experiment, we expert data, we fit a state-space projection that preserves investigate the projection matrix learned at the start of the information that makes expert behavior recognizable. The algorithm. Figure 7 gives a representative example of a projection helps discover better behaviors by ensuring that projection learned for an $\mathrm{ENC}(3)$ run. In the diagram, each skills similar to the expert are distinguishable from ranbar indicates the impact each feature of the state space has on domly initialized skills. We show the applicability of our the final embedding. The orange bar highlights the feature approach in a variety of RL tasks, ranging from a simple corresponding to linear torso velocity in the x-direction, i.e. 2 D point maze problem to continuous control locomotion. the direction in which the extrinsic objective rewards an For future work, we aim to improve the embedding scheme agent for running in. All the bars to the left correspond to of the state projection to be suitable for a wider range of joint configurations, link orientations, and all the bars to the environments. right correspond to other velocities.

The feature for velocity in the target direction is well repre- Acknowledgment. We would like to thank Kerstin Bach sented. However, so are the features for the 8 joint velocities and Rudolf Mester for their useful feedback. (8 rightmost bars in each group). Since it is a lot easier to move a single joint than to coordinate all of them for locomotion, the algorithm might more easily converge to this

\section{References} strategy than figure out a way to walk. Moreover, because Agakov, D. B. F. The im algorithm: a variational approach the projection mixes features for movement of single joints with features for locomotion of the entire body, it becomes to information maximization. Advances in neural information processing systems, 16:201, 2004. more difficult for the classifier to distinguish the two. For instance, an ant that figures out how to walk may (in the Brockman, G., Cheung, V., Pettersson, L., Schneider, J., projected space) look similar to one that only twitches some Schulman, J., Tang, J., and Zaremba, W. Openai gym. of its joints. arXiv preprint arXiv: $1606.01540,2016 .$

Campos, V., Trott, A., Xiong, C., Socher, R., Giro-i Nieto, X., and Torres, J. Explore, discover and learn: Unsupervised discovery of state-covering skills. In Interna- tional Conference on Machine Learning, pp. 1317-1327. Oudeyer, P.-Y. and Kaplan, F. What is intrinsic motivation? PMLR, $2020 .$ a typology of computational approaches. Frontiers in neurorobotics, 1:6, $2009 .$ Eysenbach, B., Gupta, A., Ibarz, J., and Levine, S. Diversity is all you need: Learning skills without a reward function. Pathak, D., Agrawal, P., Efros, A. A., and Darrell, T. In International Conference on Learning Representations, Curiosity-driven exploration by self-supervised predic$2018 .$ tion. In International Conference on Machine Learning, pp. 2778-2787. PMLR, $2017 .$ Fu, J., Luo, K., and Levine, S. Learning robust rewards with adverserial inverse reinforcement learning. In Interna- Ross, S., Gordon, G. J., and Bagnell, J. A. A reduction of imitation learning and structured prediction to no-regret tional Conference on Learning Representations, $2018 .$ online learning, 2011 . Gregor, K., Rezende, D. J., and Wierstra, D. Variational Salge, C., Glackin, C., and Polani, D. Empowerment-an intrinsic control. arXiv preprint arXiv: 1611.07507, $2016 .$ introduction. In Guided Self-Organization: Inception, pp. Haarnoja, T., Zhou, A., Abbeel, P., and Levine, S. Soft actor67-114. Springer, 2014 . critic: Off-policy maximum entropy deep reinforcement Schulman, J., Wolski, F., Dhariwal, P., Radford, A., and learning with a stochastic actor. In International ConKlimov, O. Proximal policy optimization algorithms. ference on Machine Learning, pp. 1861-1870. PMLR, arXiv preprint arXiv: $1707.06347,2017 .$ $2018 .$ Sharma, A., Gu, S., Levine, S., Kumar, V., and Hausman, Hinton, G. E. and Salakhutdinov, R. R. ReducK. Dynamics-aware unsupervised discovery of skills. In ing the Dimensionality of Data with Neural NetInternational Conference on Learning Representations, works. Science, 313(5786):504-507, July 2006 . $2019 .$ ISSN 0036-8075, 1095-9203. doi: $10.1126 /$ science. 1127647. URL https://science.sciencemag Sutton, R. S. and Barto, A. G. Reinforcement learning: An org/content/313/5786/504. Publisher: Ameriintroduction. MIT press, $2018 .$ can Association for the Advancement of Science Section: Report.

Jaderberg, M., Mnih, V., Czarnecki, W. M., Schaul, T., Leibo, J. Z., Silver, D., and Kavukcuoglu, K. Reinforcement learning with unsupervised auxiliary tasks. arXiv preprint arXiv: $1611.05397,2016 .$

Jang, E., Gu, S., and Poole, B. Categorical reparameterization with gumbel-softmax. 112016 .

LeCun, Y., Bengio, Y., and Hinton, G. Deep learning. nature, $521(7553): 436-444,2015$.

Li, K., Gupta, A., Pong, V., Reddy, A., Zhou, A., Yu, J., and Levine, S. Reinforcement learning with bayesian classifiers: Efficient skill learning from outcome examples. Deep RL Workshop, NeurIPS 2020, 2020 .

Lillicrap, T. P., Hunt, J. J., Pritzel, A., Heess, N., Erez, T., Tassa, Y., Silver, D., and Wierstra, D. Continuous control with deep reinforcement learning. URL http: / /arxiv.org/abs/1509.02971

Mnih, V., Kavukcuoglu, K., Silver, D., Rusu, A. A., Veness, J., Bellemare, M. G., Graves, A., Riedmiller, M., Fidjeland, A. K., Ostrovski, G., et al. Human-level control through deep reinforcement learning. nature, $518(7540)$ : 529-533, 2015. 

a hyperbolic tangent function, similar to Haarnoja et al. 2018).

3. The policy is updated by climbing the gradient of the minimum of the two Q functions (DDPG-style (Lillicrap et al.)).

$$
J\left(\pi_{\theta}\right)=\min _{i \in\{1,2\}} Q_{\theta}^{i}\left(\pi_{\theta}(a \mid s)\right)+\alpha \mathcal{H}\left(\pi_{\theta}(a \mid s)\right)
$$

This requires that the actions sampled from the policy are differentiable. Each gaussian component of the mixture is reparametrized the standard way, and the mixture is reparametrized with Gumbel-Softmax (Jang et al. 2016).

4. $Q_{\theta}^{1,2}$ is trained by descending on the squared temporal difference (TD) errors generated by the minimum of the target networks $Q_{\theta^{\prime}}^{1} \& Q_{\theta^{\prime}}^{2}$

$$
\begin{aligned}
T D\left(s, a, r, s^{\prime}\right) &=Q_{\theta}^{1,2}(s, a)-r-\gamma(\\
& \min _{i \in\{1,2\}} Q_{\theta^{\prime}}^{i}\left(s^{\prime}, \pi_{\theta}\left(a^{\prime} \mid s^{\prime}\right)\right)+\alpha \mathcal{H}\left(\pi_{\theta}\left(a^{\prime} \mid s^{\prime}\right)\right) \\
)
\end{aligned}
$$

\section{B. Additional Experimental Details}

This appendix extends 5 with additional plots and commentary. Figure $8.10$ show maximum, average and minimum return for the three environments.

\section{Implementation Details}

Conceptually, our skill-discovery algorithm is the same as DIAYN (Eysenbach et al. 2018). There are, however, a few implementation differences that we empirically found to work just as well. Below follows a brief rundown of the key implementation details of the algorithm used in the documented experiments.

1. Two Q-functions $Q_{\theta}^{1}(s, a) \& Q_{\theta}^{2}(s, a)$ are used, both with target clones $Q_{\theta^{\prime}}^{1} \& Q_{\theta^{\prime}}^{2}$ that are continuously updated with polyak averaging. Both Q-functions attempt to predict the same quantity:

$$
\begin{aligned}
Q_{\theta}^{1,2}\left(s_{t}, a_{t}\right) &=\mathbb{E}_{s, a \sim \pi_{\theta}}\left[r\left(s_{t}, a_{t}\right)\right.\\
&\left.+\sum_{t^{\prime}>t} \gamma^{t^{\prime}-t}\left(\alpha \mathcal{H}\left(a_{t^{\prime}}\right)+r\left(s_{t^{\prime}}, a_{t^{\prime}}\right)\right)\right]
\end{aligned}
$$

2. The policy distribution is a mixture of Gaussians with four components. The policy network predicts the mixture logits, as well as the means and log standard deviations of the Gaussians. The output is squashed through 

Figure 8: Maximum, average, and minimum return (computed over all 50 skills) for HalfCheetah-v2 during training. Shaded areas correspond to $\pm$ standard deviation across 5 random seeds.

Figure 9: Maximum, average, and minimum return (computed over all 50 skills) for Hopper-v2 during training. Shaded areas correspond to $\pm$ standard deviation across 5 random seeds.

Figure 10: Maximum, average, and minimum return (computed over all 50 skills) for Ant-v2 during training. Shaded areas correspond to $\pm$ standard deviation across 5 random seeds.
% \input{contents/3_background}
\section{Proposed Method}
The main idea of our approach is to combine the learned skills to utilize for downstream tasks. 
We gain skills in the pretrain phase following DIAYN \cite{eysenbach2018diversity} framework which is used as a general skill discovery baseline.
In DIAYN, they only used one best performing skill at the finetuning phase.
However, it is often expensive to know which skill is best-performing and loses the possibility of gaining something from another independent skills. 
As a result, we propose several methods to combine skills $\pi(a|s,\psi(z))$.

We consider $[s,\psi(z)]$ to be a state representation from the view of skill $z$.
In other words, as many representations as the number of skills are created for one state.
We propose sample efficient but robust skill combining method.
% In this formulation, state representations are \emph{mixed} over the skills by the controller $p(z|s)$.
% As above, we may replace $p(z|s)$ to $p(z)$ for simplifying.
% In this case, we only need an additional weights on the simplex $w \in \Delta^k$ for $\mathbb{E}_{z \sim p(z)}[s,z] = f_{\psi}(\sum_i w_i[s,z_i])$. 

\subsection{State-agnostic perspective fusion}
We note that the skill $z$ is a $k$-dimensional discrete random variable sampled from the distribution $p(z)$.
This is same as to making a one-hot vector by filling in some positions with 0 vectors as many as $k$ predefined.
Then we hypothesize we could learn to use all skills if we fill all the places with 1's.
This is because skill implemented as a one-hot vector is just a policy that determines which of the neural net weights to activate.
Therefore, if all positions are filled with 1, all weights can be activated and more diverse representations can be obtained.
This is interpreting the same state from different perspectives through skill.
Vanilla DIAYN proposed to attain skill vector by sampling; $z \sim p(z|s)$
The example of $z=[1, 0, 0]$. Here we propose two ways to get these different persepctives. $\psi(z)$
The final policy will be $\pi(a|s,\psi(z))$. For $\psi$, we propose two ways.
First, we propose same-weight policy $\psi(z)=[\frac{1}{k}, \frac{1}{k}, \frac{1}{k}]$.

Second is simple parametric weight policy  $\psi(z)=[w_1, w_2, w_3]$.
The input state will be transformed into using these parameters. 
% $[s',z_i'] = \sum_{i}^{d}w_i [s,z_i]/d$
Then, the transformed state will be fed into policy network to generate an action.
% $a \sim g(\phi_i(s)), \text{where} \phi_i=f([s',z_i'])$
$a \sim \pi(s, [w_1,w_2,w_3])$





\subsection{State-aware perspective fusion}
In the above method, skill weight was determined without considering incoming state.
However, if the skill weight is changed adaptively according to the state, better performance may be achieved.
DIAYN module was a skill classifier in the pretraining phase, outputting which skill was in charge for the input state.
The final output of DIAYN is logit, which could be easily trasferred to probability when feeded in softmax layer.
We utilize this probability as a importance weight for skill. During the pretraining process, 
diayn $q_{\phi}(z|s)$ predicted skill through the state and we think this pretrain task help diayn module to learn combine skills  to incoming state.
Therefore, we propose to use DIAYN learned during pretraining as skill weight predictor.
$\pi(a|s,z)$ now transforms to equation \ref{eqn:daiyn as weight predictor}.

% We decompose a skill-conditioned policy to $\pi(a|s,z) = f_{\psi}(g_{\phi}(s, z))$ where $f_{\psi}$ is a linear classifier and $g_{\phi}$ is the rest.
% Then, we write a skill-agnostic policy as
\begin{equation}
\label{eqn:daiyn as weight predictor}
    % \tilde{\pi}(a|s) = f_{\psi}(\mathbb{E}_{z \sim p(z|s)}[g_{\phi}(s,z)])
    \pi(a|s,q_{\phi}(z|s))
\end{equation}


\subsection{Analyzation}
TODO: analyze a trajectory and the representations of each state



\section{Experiments on unsupervised RL benchmark}

Through the experiment, we show sample efficient training is possible only by adding trivial number of parameters(=skill-dim).
We experimented our methods on URLB ($https://github.com/rll-research/url_benchmark$) where unsupervised RL methods could be compared fairly and easily.

There are two phases in URLB.
First phase is a reward-free pretrain phase and the other is finetuning phase with explicit rewards.
In a reward-free environment, methods such as DIAYN tries to train the agent to learn something meaningful using intrinsic reward.
Through this pretrain phase, the agent learns some behavior and we call this behavior as skill.
Then we use these skills the get a higher reward faster in finetuning. Our method suggests several methods to combines these skills together and shows great performance.

DIAYN is fixed as the pretrain method and finetuning is performed in various ways.
To follow the comparison introduced in the CIC paper, 2 million steps are pretrained and then 100,000 steps finetuning is performed.
Our method outperformed the other pretrain-finetune methods in 12 environments and the result is summarized on table \cref{finetuning result}.
% It is noteworthy that the simplest state agnostic skill weight method among our proposed methods obtained the best performance. 
State agnostic methods showed faster learning, but state aware method achieved better result at the end.



\subsection{State-agnostic perspective fusion}
We introduced two state-agnostic perspective combining methods.
First is to assign same weights to each perspective, and the second is to introduce several parameters which counts to skill dim.
While the first method doesn't require any additional parameters, the second methods use several parameters are used to weight each skill using additional parameters as large as the skill dimension.
The parameters are trainined in end-to-end manner using DDPG which is a default finetuning method in URLB.
These parameters serve to fuse the state viewed from various perspectives to help the agent obtain the optimal reward in a faster way.

We compare the results of using our method and not using our method during finetune when the pretrain method is both DIAYN.
The default DIAYN method is blue line on \cref*{state agnostic results}, and all other methods exceed this baseline.

We believe this result came about for two reasons.
First, the default DIAYN finetune uses only one skill per episode.
This has disadvantage of not being able to utilize other skills learned during pretraining.
Secondly, the sampled skill is not optimal with high probability.
Since there's no way default DIAYN finetuning know which skill is good for the downstream task,it samples the skill uniformly.
And this uniform sampling of course highly not optimal.
This may be useful if interpreted as a method of obtaining a general agent in multi-task learning,
but it is inappropriate as an approach that aim to achieve maximum performance in a specific goal-oriented downstream task.
In fact, using only one fixed skill which is chosen at random far exceeds default DIAYN implemented in URLB.(Red line)
This reveals that randomly changing skill at finetuning phase harms a performance a lot.


\subsubsection{Same skill weight and Simple learnable skill weight}
While DIAYN and fixed one weight method uses only one skill, our proposed method utilizes all skills.
Same skill weight assigns same importance to all skills and simple weight scheme learns to control skill weights to achieve better performance.
Notable thing is that parameter-less same skill weight scheme outperformed all other methods.
This is because just to open a way to use all the skills was enough to achieve good performance
and it seems that changing skill weight is also possible by adjusting the weight of the latter layers of neural net.
In other words, 
% In \cref*{final_skill_weight}, we can check the agent use skills near evenly.
% This allows for faster and better performance, as shown in the \cref{finetuning result}.


% \begin{tabular*}{\textwidth}{lcccr}
%   \begin{table}[t]
%     \caption{Performance at finetuning}
%     \label{finetuning result}
%     \vskip 0.15in
%     \begin{center}
%     \begin{small}
%     \begin{sc}
%     \begin{tabular*}{\textwidth}{c @{\extracolsep{\fill}} ccccc}
%     \toprule
%     Domain & Task & BASE DIAYN & Other best & Ours \\
%     \midrule
%     Walker & Flip    &  \\
%      & Run    & 158$\pm$8 & 486$\pm$25 & 542\\
%      & Stand    &  \\
%      & Walk    &  \\
%     Quadruped & Jump    &  \\
%      & Run    &  \\
%      & Stand    &  \\
%      & Walk   &  \\
%     Jaco & Reach bottom left    &  \\
%      & Reach bottom right   &  \\
%      & Reach top left    & \\
%      & Reach top right   &  \\
%     \bottomrule
%   \end{sc}
% \end{small}
% \end{center}
% \vskip -0.1in
% \end{table}
% \end{tabular*}


  \begin{figure}[ht]
    \vskip 0.2in
    \begin{center}
    \centerline{\includegraphics[width=\columnwidth]{Figures/multiple_seed_state_agnostic_methods.png}}
    \caption{Finetuning results of state agnostic skill weight methods. Same skill importance method achived the best result.}
    \label{state agnostic results}
    \end{center}
    \vskip -0.2in
    \end{figure}
Looking at the final learned ${w_i}$, it can be seen that all perspectives are used near evenly.


\subsubsection{Fixed one skill}
% \subsubsection{Sampling skill(DIAYN)}

\subsubsection{Zero skill weight}
We came to a question when we saw that skill weight was evenly distributed.
If we take average of ${[s,z]}$ vectors, isn't it the same as just having state without skill?
This is because the more evenly the importance is distributed, the closer the skill vector is to 0 vector in one-hot vector. Figure needed?
This is because, when concating, the mean of state vector part will be same as the original value,
but the skill vector part will be a near-uniform distributed vector of about $\frac{1}{skilldim}$.
This makes it doubtful whether the skill vector part has any meaning.

If weighted skill vector has really no meaning, then the agent without skill should outperform our method.
Therefore, we compare our method with DDPG + DDPG agent. The result is that our method is better.

This reveals that the skill allows it to learn some meaningful behavior.
Maybe the skill distribution looks like a flat, meaningless vector to our eyes, but the slight difference is meaningful.
If weighted skill vector has really no meaning, then the agent without skill should outperform our method.
Therefore, we compare our method with DDPG + DDPG agent. The result is that our method is better.

This reveals that the skill allows it to learn some meaningful behavior.
Maybe the skill distribution looks like a flat, meaningless vector to our eyes, but the slight difference is meaningful.

\subsection{DIAYN as skill weight predictor}
The second is to use the DIAYN module used in the pretrain phase as a skill weight predictor.
Also, we train DIAYN in end to end manner using DDPG.

 But the performance was not good. This may be because the pretrain task does not perform a very good weight initializer role,
 or it may be because it is not good to change the weight of skill according to the state.

\subsubsection{Weight transfer from pretrained DIAYN}

\cref*{diayn-as-skill-weight}
In the case of simple weight, skill weight is determined regardless of the incoming state.
However, using the DIAYN module as skill weight predictor creates an association between state and skill weights.
With the transfered weight from pretrained phase, DIAYN module learns to output the skill weight in end to end manner.
However, the learning results were not good.
There was a part where the performance deteriorated significantly for a certain period.

\begin{figure}[ht]
  \vskip 0.2in
  \begin{center}
  \centerline{\includegraphics[width=\columnwidth]{Figures/fair_weight_and_diayn_as_weight_predictor.png}}
  \caption{Historical locations and number of accepted papers for International
  Machine Learning Conferences (ICML 1993 -- ICML 2008) and International
  Workshops on Machine Learning (ML 1988 -- ML 1992). At the time this figure was
  produced, the number of accepted papers for ICML 2008 was unknown and instead
  estimated.}
  \label{diayn-as-skill-weight}
  \end{center}
  \vskip -0.2in
  \end{figure}

\subsubsection{Train from scratch}
Although it recovered later, but even after recovery, it did not show imporoved performance than the baseline and didn't gain sample efficiency from weight transfer.
We thought that DIAYN weight would be a good initializer as a skill weight predictor, the above experiment reveales we are wrong.
Instead, we randomly init the weight with the intention of using it only as a module to determine the appropriate skill weight depending on the state.
The results were rather better than transferred weight.
This shows that the pretrain task was not very helpful in predicting how to fuse the skill together.




\subsection{MultiHeadAttention to attain several skill weight}
Third, we use MultiHeadAttention to have multiple the skill weights.
The result was fine.
However, there is not much gain compared to Simple Weight.
In this case, using simple weights is parameter efficient.

\subsubsection{self attention}
\subsubsection{use state as query attention}






\section{Conclusion}


In this paper, we propose a method to enable faster and better learning by combining the skills learned in the pretrain phase.
The interesting thing is that it is better to have a skill weight regardless of the state than to have a different skill weight for each state.
Based on these results, we want to find a way to better utilize skill combinations such as attention in the future.



% \subsection{Attention}
\begin{figure}[hb]
  \vskip 0.2in
  \begin{center}
  \centerline{\includegraphics[width=\columnwidth]{Figures/attention_on_walker_run.png}}
  \caption{This result was lost}
  \label{attention-on-walker-run}
  \end{center}
  \vskip -0.2in
  \end{figure}



% Camera-ready copies should have the title of the paper as running head
% on each page except the first one. The running title consists of a
% single line centered above a horizontal rule which is $1$~point thick.
% The running head should be centered, bold and in $9$~point type. The
% rule should be $10$~points above the main text. For those using the
% \textbf{\LaTeX} style file, the original title is automatically set as running
% head using the \texttt{fancyhdr} package which is included in the ICML
% 2022 style file package. In case that the original title exceeds the
% size restrictions, a shorter form can be supplied by using

% \verb|\icmltitlerunning{...}|

% just before $\mathtt{\backslash begin\{document\}}$.
% Authors using \textbf{Word} must edit the header of the document themselves.





% Acknowledgements should only appear in the accepted version.
% \section*{Acknowledgements}

% \textbf{Do not} include acknowledgements in the initial version of
% the paper submitted for blind review.

% If a paper is accepted, the final camera-ready version can (and
% probably should) include acknowledgements. In this case, please
% place such acknowledgements in an unnumbered section at the
% end of the paper. Typically, this will include thanks to reviewers
% who gave useful comments, to colleagues who contributed to the ideas,
% and to funding agencies and corporate sponsors that provided financial
% support.


% In the unusual situation where you want a paper to appear in the
% references without citing it in the main text, use \nocite
\nocite{langley00}

\bibliography{example_paper}
\bibliographystyle{icml2022}


%%%%%%%%%%%%%%%%%%%%%%%%%%%%%%%%%%%%%%%%%%%%%%%%%%%%%%%%%%%%%%%%%%%%%%%%%%%%%%%
%%%%%%%%%%%%%%%%%%%%%%%%%%%%%%%%%%%%%%%%%%%%%%%%%%%%%%%%%%%%%%%%%%%%%%%%%%%%%%%
% APPENDIX
%%%%%%%%%%%%%%%%%%%%%%%%%%%%%%%%%%%%%%%%%%%%%%%%%%%%%%%%%%%%%%%%%%%%%%%%%%%%%%%
%%%%%%%%%%%%%%%%%%%%%%%%%%%%%%%%%%%%%%%%%%%%%%%%%%%%%%%%%%%%%%%%%%%%%%%%%%%%%%%
\newpage
\appendix
\onecolumn
\section{You \emph{can} have an appendix here.}

%% TABLE %%%%%%%%%%%%%%%%%%%%%%%%%%%%%
\begin{table*}[t]
    \caption{Result of fine-tuning for $1 \times 10^5$ frames after pre-training for $2 \times 10^6$ frames.}
    \label{table:result_urlb}
    \vskip 0.15in
    \begin{center}
    \begin{small}
    \begin{sc}
    \begin{tabular}{lccccc}
    % \begin{tabular*}{\textwidth}{c @{\extracolsep{\fill}} ccccc}
    \toprule
    Domain & Task & Expert & Other best & Original DIAYN & Ours (same weight) \\
    \midrule
    Walker & Flip  & 799  & 515$\pm$17 &  381$\pm$17   & \textbf{658$\pm$51}\\
           & Run   & 796  & 439$\pm$34 &  242$\pm$11   & \textbf{537$\pm$22}\\
           & Stand & 984  & 923$\pm$9  &  860$\pm$26   & \textbf{936$\pm$11} \\
           & Walk  & 971  & 828$\pm$29 &  661$\pm$26   & \textbf{917$\pm$23}\\
    Quadruped & Jump  & 888  & 590$\pm$33  &  578$\pm$46  & \textbf{645$\pm$20} \\
           & Run   & 888     & 465$\pm$37  &  415$\pm$28  & \textbf{558$\pm$43} \\
           & Stand & 920     & \textbf{840$\pm$33}  &  706$\pm$48  & 719$\pm$158 \\
           & Walk  & 866     & 721$\pm$56  &  406$\pm$64  & \textbf{845$\pm$74}  \\
    Jaco & Reach bottom left   & 193 & 134$\pm$8 & 17$\pm$5   & \textbf{136$\pm$36}\\
           & Reach bottom right& 203 & \textbf{122$\pm$4} & 31$\pm$4   & 119$\pm$39\\
           & Reach top left    & 191 & 124$\pm$20 & 11$\pm$3   & \textbf{127$\pm$9}\\
           & Reach top right   & 223 & \textbf{140$\pm$7}  & 19$\pm$4   & 138$\pm$42\\
    \bottomrule
    \end{tabular}
    \end{sc}
    \end{small}
    \end{center}
    \vskip -0.1in
    \end{table*}
   %% TABLE %%%%%%%%%%%%%%%%%%%%%%%%%%%%%
%%%%%%%%%%%%%%%%%%%%%%%%%%%%%%%%%%%%%%%%%%%%%%%%%%%%%%%%%%%%%%%%%%%%%%%%%%%%%%%
%%%%%%%%%%%%%%%%%%%%%%%%%%%%%%%%%%%%%%%%%%%%%%%%%%%%%%%%%%%%%%%%%%%%%%%%%%%%%%%


\end{document}


% This document was modified from the file originally made available by
% Pat Langley and Andrea Danyluk for ICML-2K. This version was created
% by Iain Murray in 2018, and modified by Alexandre Bouchard in
% 2019 and 2021 and by Csaba Szepesvari, Gang Niu and Sivan Sabato in 2022. 
% Previous contributors include Dan Roy, Lise Getoor and Tobias
% Scheffer, which was slightly modified from the 2010 version by
% Thorsten Joachims & Johannes Fuernkranz, slightly modified from the
% 2009 version by Kiri Wagstaff and Sam Roweis's 2008 version, which is
% slightly modified from Prasad Tadepalli's 2007 version which is a
% lightly changed version of the previous year's version by Andrew
% Moore, which was in turn edited from those of Kristian Kersting and
% Codrina Lauth. Alex Smola contributed to the algorithmic style files.
